\documentclass[10pt]{beamer}

\usepackage[brazilian]{babel}
\usepackage[utf8]{inputenc}
\usepackage{fancybox}

\usetheme{Boadilla}
\usecolortheme{crane}

\title{\LaTeX}
\subtitle{Um curso de \LaTeX\ em forma de apresentação\footnote{Tradução e adaptação do original de Rainer Rupprecht}}
\institute{IFRN - Campus Pau dos Ferros}
\author{Prof. Diego Cirilo}
\date{Fev. 2013}
\titlegraphic{\includegraphics[height=10mm]{img/ifrn}}

\begin{document}

\frame{\titlepage}

\section{Introdução}
\begin{frame}
  \frametitle{O \LaTeX}
  \begin{description}
    \item[\TeX]{sistema de formatação de texto desenvolvido por
        \textsc{Donald E. Knuth} (Stanford University) para criar documentos
        bonitos, especialmente contendo matemática. \TeX\ é um software gratuito.}
    \item[\LaTeX]{\TeX -macroprocessador escrito por \textsc{Leslie
        Lamport}, que implementa uma linguagem de marcação (similar: HTML, XML).
      Usuários podem se ocupam da estrutura ao invés da formatação.}
    \item[WYSIWYG]{What You See Is What You Get, comportamento de editores de texto
      como Word, Writer, etc.}
  \end{description}
\end{frame}

\subsection{Vantagens e Desvantagens}

\begin{frame}
  \frametitle{Vantagens}
  \begin{itemize}
     \item{Vários estilos profissionais disponíveis.
         Mudar entre estilos requer pouco esforço e a consistência é mantida.}
     \item{Formatação matemática de alta qualidade.}
     \item{Poucos comandos para definir a estrutura de texto, não é necessário se
       preocupar com tipografia ou \textit{layout}.}
     \item{Documentos e estruturas científicas podem ser criados rapidamente:
         \begin{itemize}
           \item{bibliografia}
           \item{índice}
           \item{referências cruzadas}
           \item{índice remissimo, lista de figuras, tabelas, etc.}
           \item{...}
         \end{itemize} }
     \item{Independente de sistema operacional.}
     \item{Arquivos de texto simples.}
     \item{Software Livre!}
  \end{itemize}
\end{frame}

\begin{frame}
  \frametitle{Desvantagens}
  \begin{itemize}
    \item{Curva de aprendizado}
    \item{Grandes alterações no layout exigem um bom conhecimento da ferramenta.}
    \item{É impossível usar outra ferramenta depois do \LaTeX.}
  \end{itemize}
\end{frame}

\subsection{Fluxo de operação}
\begin{frame}
  \frametitle{Fluxo de Operação}
  \begin{itemize}
    \item O \LaTeX\ é escrito em texto puro, com os comandos ou \textit{tags} para designar a formatação.
    \item Esse texto é então "compilado", ou seja, processado e então é gerado o arquivo final, comumente PDF.
    \item Portanto para escrever \LaTeX, precisamos de duas ferramentas:
      \begin{itemize}
        \item Editor de Texto
        \item Compilador
      \end{itemize}
  \end{itemize}
\end{frame}

\subsection{Editor de Texto}
\begin{frame}
  \frametitle{Editor de Texto}
  \begin{itemize}
    \item Usado para escrever o código \LaTeX\ em si.
    \item Pode ser tão simples quanto o Bloco de Notas do Ruindows, ou
    \item Ser específico para \LaTeX, como o \TeX Maker, etc.
    \item A vantagem dos editores específicos é oferecer funções de compilação automática, \textit{highlight}, \textit{templates}, etc.
  \end{itemize}
\end{frame}

\section{Estrutura de um documento \LaTeX}

\begin{frame}[fragile]
  \frametitle{Estrutura de um documento \LaTeX}
  \center{\footnotesize 
    \begin{verbatim}
    % Um comentário... é ignorado pelo compilador

    \documentclass[options]{style} %define o tipo ou "estilo" de documento

    \usepackage[latin1]{inputenc}  %pacotes são adicionados nessa região para as
    \usepackage[T1]{fontenc}       %mais diversas funções extras.

    \author{}
    \title{}
    \date{}

    \begin{document}               %início do documento
    \maketitle                     %cria cabeçalho ou folha de rosto/títulos

    \chapter{}                     %secionamento
    ...
    \end{document}                 %fim do documento
    \end{verbatim}
  }
\end{frame}

\subsection{Classes}
\begin{frame}[fragile]
  \frametitle{Classes de Documentos}
  \center{
    \begin{tabular}{l|l}
      \LaTeX & propósito \\
      \hline
      article & artigos, relatórios curtos \\
       report & textos longos com vários cápitulos, ex. teses. \\
       book   & livros \\
       letter & cartas \\
       beamer & apresentações de slides \\
      sciposter & posters para conferências \\
    \end{tabular}
  }
  \vskip 10pt
  Além desses, ainda há outros pacotes para estilos específicos, como o abn\TeX2, ou estilos próprios de
  congressos e sociedades, como o IEEEtran.
\end{frame}

\begin{frame}
  \frametitle{Opções de classe}
  \begin{description}
    \item[Tamanho da Fonte]{10pt | 11pt | 12pt...}
    \item[Tamanho do Papel]{a4paper | legalpaper...}
    \item[Tipos de equações]{fleqn, leqno }
    \item[Título]{titlepage |  notitlepage}
    \item[Colunas]{onecolumn | twocolumn}
    \item[Impressão]{oneside | twoside}
  \end{description}
\end{frame}

\section{Comandos Básicos}

\subsection{Comandos Especiais}

\begin{frame}[fragile]
  \frametitle{Caracteres Especiais}
  Alguns caracteres têm funções especiais em \TeX, portanto para utiliza-los, é necessário usar comandos:

  \vspace{5mm}
  \begin{tabular}{l|l|l}
    $\backslash$~~   & iniciar comando        & \verb+$\backslash$+    \\
                     &                      & notar: \verb+\\+ = nova linha \\
    \$               & entrar em modo matemático    & \verb+\$+              \\
    \&               & tabulação            & \verb+\&+              \\
    \%               & comentário & \verb+\%+              \\
    \#               &                      & \verb+\#+              \\
    \textasciitilde  &                      & \verb+\textasciitilde+ \\
    \textbar         & linhas verticais & \verb+\textbar+        \\
    \_               & subscrito     & \verb+\_+              \\
    \textasciicircum & sobrescrito    & \verb+\textasciicircum+\\
    \{ \}            & limites de comando    & \verb+\{ \}+           \\
    $[\ ]$           & argumentos opcionais   & \verb+$[ ]$+           \\
    `` ''            & aspas      & \verb+`` ''+           \\
    $> <$            & tabulação              & \verb+$> <$+           \\
  \end{tabular}
\end{frame}

\subsection{Hífens}

\begin{frame}[fragile]
  \frametitle{Hífens}
  Uma curta linha vertical pode significar várias coisas,
  dependendo da espessura e comprimento...

  \vspace{5mm}
  \begin{tabular}{lll}
    guarda-chuva           & \verb+guarda-chuva+           \\
    10--18 horas   & \verb+10--18 horas+   \\
    sim -- não acha? & \verb+sim -- não acha?+ \\
    sim---ou não?     & \verb+sim---ou não?+     \\
    0, 1 e --1     & \verb+0, 1 e --1+     \\
  \end{tabular}
\end{frame}

\section{Estruturando o Texto}

\subsection{Secionamento}

\begin{frame}[fragile]
  \frametitle{Comandos para secionamento}
  \begin{itemize}
    \item{{$\backslash$part\{\}}}
    \item{{$\backslash$chapter\{\}}}
    \item{$\backslash$section\{\}}
    \item{$\backslash$subsection\{\}}
    \item{$\backslash$subsubsection\{\}}
    \item{$\backslash$paragraph\{\}}
  \end{itemize}
\end{frame}

\begin{frame}
  \frametitle{Dividindo longos documentos}
  É uma boa prática dividir longos arquivos em arquivos menores, e com \LaTeX\ isso é possível.
  Por exemplo, cada capítulo em um arquivo, etc.
    \begin{description}
    \item[$\backslash$input\{\}]{lê arquivo diretamente}
    \item[$\backslash$include\{\}]{equivalente a
        $\backslash$clearpage $\backslash$input\{\}
      $\backslash$clearpage}
    \item[$\backslash$includeonly\{\}]{usado para limitar os arquivos $\backslash$included }
  \end{description}
\end{frame}

\begin{frame}
  \frametitle{Sumarização}
    \begin{itemize}
        \item{$\backslash$tableofcontents}
        \item{$\backslash$listoffigures}
        \item{$\backslash$listoftables}
        \item{}
        \item{$\backslash$bibliographystyle\{plainnat\}}
        \item{$\backslash$bibliography\{references\}}
        \item{}
        \item{$\backslash$printindex}
    \end{itemize}
\end{frame}

\subsection{Fontes}

\begin{frame}
  \frametitle{Ênfase}
    \begin{description}
       \item[$\backslash$textit\{\}]{itálico, usado para palavras estrangeiras, nomes de espécies, etc: \textit{Staph.
           aureus}}
       \item[$\backslash$textsl\{\}]{\textsl{inclinado}}
       \item[$\backslash$emph\{\}]{:
         ênfase, isso \emph{é} importante}
       \item[$\backslash$textsc\{\}]{small caps: \textsc{Neil Armstrong} primeiro homem a pisar na lua (ou não).}
       \item[$\backslash$textbf\{\}]{negrito: totalmente \textbf{desnecessário}.}
       \item[$\backslash$texttt\{\}]{fonte monospace, usada para código fonte ou URLs: \texttt{http://www.ifrn.edu.br/}}
    \end{description}
\end{frame}

\begin{frame}[fragile]
  \frametitle{Tamanho de Fontes}
    \center{\begin{tabbing} \texttt{xfootnotesizexx}\= und dann der Text
        \kill
        \verb|\tiny|         \> \tiny         fonte microscópica           \\
        \verb|\scriptsize|   \> \scriptsize   fonte muito pequena (subscrito)\\
        \verb|\footnotesize| \> \footnotesize fonte bem pequena (nota de rodapé)      \\
        \verb|\small|        \> \small        fonte pequena                 \\
        \verb|\normalsize|   \> \normalsize   fonte normal                \\
        \verb|\large|        \> \large        fonte grande                 \\
        \verb|\Large|        \> \Large        fonte maior                \\
        \verb|\LARGE|        \> \LARGE        fonte maior ainda            \\[3pt]
        \verb|\huge|         \> \huge         fonte gigantesca                  \\[3pt]
        \verb|\Huge|         \> \Huge         fonte colossal
    \end{tabbing}}
\end{frame}

\subsection{Listas}

\begin{frame}[fragile]
  \frametitle{Listas Simples}
    Please believe me:
    \begin{itemize}
        \item{Few swallows can turn winter into summer.}
        \item{Inside it's colder than in the night.
            \begin{itemize}
                \item{In the morning it pulls.}
                \item{At noon he pushes.}
                \item{In the evening she goes.}
                \end{itemize} }
        \item{Every nonsense must find an end.}
    \end{itemize}
    {\small \begin{verbatim}
Please believe me:
\begin{itemize}
    \item{Few swallows can turn winter into summer.}
    \item{Inside it's colder than in the night.
        \begin{itemize}
            \item{In the morning it pulls.}
            \item{At noon he pushes.}
            \item{In the evening she goes.}
        \end{itemize} }
    \item{Every nonsense must find an end.}
\end{itemize}
    \end{verbatim} }
\end{frame}

\begin{frame}[fragile]
  \frametitle{Listas Descritivas}
    Três animais que você deve conhecer:
    \begin{description}
        \item[Catita:]{Um animal pequeno.}
        \item[Rato:]{Um animal de médio porte.}
        \item[Guabiru:]{Um animal de respeito}
    \end{description}
    {\small \begin{verbatim}
    \begin{description}
        \item[Catita:]{Um animal pequeno.}
        \item[Rato:]{Um animal de médio porte.}
        \item[Guabiru:]{Um animal de respeito}
    \end{description}
\end{verbatim} }
\end{frame}

\begin{frame}[fragile]
  \frametitle{Listas Enumeradas}
    Estes são os principais pontos:
    \begin{enumerate}
        \item{primeiro item}
        \item{segundo item}
        \item{terceiro item
           \begin{enumerate}
             \item{primeiro sub-item}
             \item{segundo sub-item}
           \end{enumerate} }
    \end{enumerate}
    {\small \begin{verbatim}
    Estes são os principais pontos:
    \begin{enumerate}
        \item{primeiro item}
        \item{segundo item}
        \item{terceiro item
           \begin{enumerate}
             \item{primeiro sub-item}
             \item{segundo sub-item}
           \end{enumerate} }
    \end{enumerate}
\end{verbatim} }
\end{frame}

\subsection{Tabelas}

%\begin{frame}[fragile]
  %\frametitle{Tabelas}
    %\begin{tabbing}
        %If \= it's raining           \\
            %\> then \= put on boots, \\
            %\>      \> take hat;     \\
            %\> else \> smile.        \\
        %Leave house.
    %\end{tabbing}
    %{\small \begin{verbatim}
%\begin{tabbing}
    %If \= it's raining           \\
        %\> then \= put on boots, \\
        %\>      \> take hat;     \\
        %\> else \> smile.        \\
    %Leave house.
%\end{tabbing}
    %\end{verbatim} }
%\end{frame}

\begin{frame}[fragile]
  \frametitle{Tabelas}
  \center{
    \begin{tabular}{|r||c|p{2.5in}|}
      \hline
      \multicolumn{3}{|c|}{\sc Histórico Silva\&\ Silva} \\
      \hline
      \hline
      \multicolumn{1}{|c||}{\bf Ano} & \bf Preço   & \multicolumn{1}{c|}{\bf Comentários} \\
      \hline
      1971 & 97--245   & Um mal ano para fazendeiros no oeste.     \\ \hline
        72 & 245--245  & Pouca produção devido à seca.  \\ \hline
        73 & 245--2001 & Ótima produção. \\ \hline
    \end{tabular}
  }
    {\scriptsize \begin{verbatim}
    \begin{tabular}{|r||c|p{2.5in}|}
        \hline
        \multicolumn{3}{|c|}{\sc Histórico Silva\&\ Silva} \\
        \hline
        \hline
        \multicolumn{1}{|c||}{\bf Ano} & \bf Preço   & \multicolumn{1}{c|}{\bf Comentários} \\
        \hline
        1971 & 97--245   & Um mal ano para fazendeiros no oeste.     \\ \hline
          72 & 245--245  & Pouca produção devido à seca.  \\ \hline
          73 & 245--2001 & Ótima produção. \\ \hline
    \end{tabular}
    \end{verbatim} }
    Nota: Esse é um péssimo exemplo de tabela!
\end{frame}

\subsection{Imagens}

\begin{frame}[fragile]
  \frametitle{Imagens}
  \center{\includegraphics[width=3cm]{img/ifrn}}

  {\footnotesize \begin{verbatim}
  \center{\includegraphics[width=3cm]{img/ifrn}}
  \end{verbatim} }
  \begin{itemize}
     \item{Requer \verb+\usepackage{graphicx}+}
     \item{Vários formatos de arquivos possíveis. Para o pdfLaTeX pdf, png, jpg. }
     \item{Argumentos opcionais: width, angle, size}
  \end{itemize}
\end{frame}

\section{Matemática}

\subsection{Ambientes matemáticos}

\begin{frame}[fragile]
  \frametitle{Matemática embutida}
    se $a$ e $b$ são os lados de um
    triangulo reto e $c$ a hipotenusa, então
    $c^2=a^2+b^2$ (Teorema de Pitágoras).

    {\small \begin{verbatim}
    se $a$ e $b$ são os lados de um
    triangulo reto e $c$ a hipotenusa, então
    $c^2=a^2+b^2$ (Teorema de Pitágoras).
    \end{verbatim} }
\end{frame}

\begin{frame}[fragile]
  \frametitle{Matemática isolada}
    se $a$ e $b$ são os lados de um
    triangulo reto e $c$ a hipotenusa, então
     \begin{equation}
        c^2=a^2+b^2
     \end{equation}
    (Teorema de Pitágoras).

    {\small \begin{verbatim}
    se $a$ e $b$ são os lados de um
    triangulo reto e $c$ a hipotenusa, então
     \begin{equation}
        c^2=a^2+b^2
     \end{equation}
    (Teorema de Pitágoras).
    \end{verbatim} }
\end{frame}

\subsection{Comandos matemáticos básicos}

\begin{frame}[fragile]
  \frametitle{Fórmulas}
    \begin{displaymath}
        x^5 \qquad \qquad x_1 \qquad \qquad \sqrt{x^2+\sqrt[3]{y}}
    \end{displaymath}
    \begin{verbatim}
    x^5     x_1       \sqrt{x^2+\sqrt[3]{y}}
    \end{verbatim}

    \begin{displaymath}
        \frac{1}{\frac{x^2+y^2+z^2}{x+y}} \qquad \qquad {n\choose {n-k}}
    \end{displaymath}
    \begin{verbatim}
    \frac{1}{\frac{x^2+y^2+z^2}{x+y}}  {n\choose {n-k}}
    \end{verbatim}

    \begin{displaymath}
        \int \limits_{-\infty}^{\infty}x^3 \qquad \qquad \sum_{i=1}^{n}a_i
    \end{displaymath}
    \begin{verbatim}
    \int \limits_{-\infty}^{\infty}x^3  \sum_{i=1}^{n}a_i
    \end{verbatim}
\end{frame}

\begin{frame}[fragile]
  \frametitle{Alinhando equações}
    \begin{align}
        f(x)                & =  \cos x \\
        f'(x)               & =  -\sin x \\
        \int_{0}^{x} f(y)dy & =  \sin x
    \end{align}
    {\small \begin{verbatim}
\begin{align}
    f(x)                & =  \cos x \\
    f'(x)               & =  -\sin x \\
    \int_{0}^{x} f(y)dy & =  \sin x
\end{align}
    \end{verbatim} }
\end{frame}

\section{Changing the layout}

\subsection{Counters and parameters}

\subsection{Distâncias}

\begin{frame}[fragile]
  \frametitle{Distância horizontal}
    Aqui temos \hspace{2cm} 2cm de distância.

    {\begin{verbatim}
    Aqui temos \hspace{2cm} 2cm de distância.
    \end{verbatim}}

    esquerda\hfill direita
    {\begin{verbatim}
    esquerda\hfill direita
    \end{verbatim}}
    \begin{tabbing}
        \texttt{xenspace}\qquad \= \kill
        \verb|\,|       \> distancia pequena \\
        \verb|\enspace| \> distância de um número \\
        \verb|\quad| \> tão largo quanto a altura de uma letra \\
        \verb|\qquad| \> o dobro do anterior \verb|\quad| \\
        \verb|\hfill| \> distância que vai de 0\\
                     \> a $\infty$ \\
    \end{tabbing}
\end{frame}

\begin{frame}[fragile]
  \frametitle{Distância Vertical}
    Aqui

    \vspace{2cm}

    há 2 cm de distância.
    \begin{verbatim}
    Aqui

    \vspace{2cm}

    há 2 cm de distância.
    \end{verbatim}

    \begin{tabbing}
        \texttt{xsmallskip}\qquad \= \kill
        \verb|\smallskip| \> cerca de $1/4$ de linha \\
        \verb|\medskip| \> cerca de $1/2$ linha \\
        \verb|\bigskip| \> cerca de 1 linha \\
        \verb|\vfill| \> distância que pode ir de 0\\
               \> a $\infty$ \\
    \end{tabbing}
\end{frame}

\subsection{Posição do texto}

\begin{frame}[fragile]
  \frametitle{Centralizando Texto}
    \begin{center}
        In\\
        the\\
        middle I don't\\
        feel\\
        so marginalized\\
    \end{center}
    \begin{verbatim}
    \begin{center}
      In\\
      the\\
      middle I don't\\
      feel\\
      so marginalized\\
    \end{center}
    \end{verbatim}
\end{frame}

\begin{frame}[fragile]
  \frametitle{Alinhamento à direita}
    \begin{flushright}
       Essa não é uma posição política
    \end{flushright}
    \begin{verbatim}
    \begin{flushright}
       Essa não é uma posição política
    \end{flushright}
    \end{verbatim}
\end{frame}

\section{Especiarias}

%\subsection{O Índice}

%\begin{frame}[fragile]
  %\frametitle{Indexing commands}
  %\begin{description}
    %\item[simple]{\verb+gnat\index{gnat}+}
    %\item[subtopics]{\verb+gnat\index{gnat!size of}+}
    %\item[page range]{\verb+\index{gnat|(}...\index{gnat|)}+}
    %\item[reference]{\verb+\index{gnat|see{mosquito}}+}
    %\item[font]{\verb+gnat\index{gnat@\textit{gnat}}+}
  %\end{description}
  %After first \LaTeX\ run, start \alert{makeindx} to sort the index.
%\end{frame}

\subsection{Bibliografia}

\begin{frame}[fragile]
  \frametitle{Bib\TeX}
  Ferramenta para organização de referências no formato ASCII. Pode ser produzida por várias ferramentas.  
  \center{\small \begin{verbatim}
@article{Alb-76,
  AUTHOR= {W.J. Albery and J.R. Knowles},
  TITLE= {Evolution of enzyme function and the
    development of catalytic efficiency},
  JOURNAL= {Biochemistry},
  VOLUME= {15},
  YEAR= {1976},
  PAGES= {5631-5640},
  ABSTRACT= {Catalytic efficiency constant kcat/Km
    defined },
  DOI= {10.1021/bi00670a032},
  LANGUAGE= {engl}
}
  \end{verbatim} }
  Similar para livros, teses, etc. No texto use $\backslash$cite\{Alb-76\}.
\end{frame}

\subsection{Apresentações de Slides}

\begin{frame}
  \frametitle{Beamer-slides}
  $\backslash$begin\{frame\} \\
  \hspace{2em}$\backslash$frametitle\{\} \\
  \hspace{2em}... \\
  $\backslash$end\{frame\}
\end{frame}

%\subsection{TeX-ing}

%\begin{frame}
  %\frametitle{\TeX ing}
  %Always use the sequence:
  %\begin{description}
    %\item[\LaTeX]{produces the necessary intermediate
        %files}
    %\item[makeindx]{sort the index}
    %\item[bibtex]{create the bibliography}
    %\item[\LaTeX]{include bibliography and index, resolve
        %cross-references}
    %\item[\LaTeX]{resolve remaining cross-references}
  %\end{description}
  %Note: using pdf\LaTeX\ instead of \LaTeX\ produces pdf-files
  %directly.
%\end{frame}

\end{document}
